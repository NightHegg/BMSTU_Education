\documentclass[a4paper]{article}

\usepackage[14pt]{extsizes} % чтобы использовать шрифт размером больше 12
\usepackage{cmap} % для кодировки шрифтов в pdf
\usepackage[T2A]{fontenc} % пакет указывает внутреннюю кодировку в системе LaTeX
\usepackage[utf8]{inputenc} % кодировка  
\usepackage[english, russian]{babel} % пакет для локализации

\usepackage{graphicx} % для вставки картинок
\usepackage{amssymb,amsfonts,amsmath,amsthm} % математические дополнения от АМС
\usepackage{indentfirst} % отделять первую строку раздела абзацным отступом тоже
\usepackage{makecell} % для создания таблиц
\usepackage{multirow} % для продвинутых таблиц
\usepackage{setspace} % для изменения междустрочного интервала
\usepackage{ulem} % подчеркивания

\usepackage{listings} %для вставки кода
\usepackage[linesnumbered,lined,boxed,commentsnumbered]{algorithm2e}

\usepackage[left=20mm, top=15mm, right=15mm, bottom=15mm, nohead, footskip=10mm]{geometry} % настройки полей документа

\linespread{1.3} % полуторный интервал
 
\begin{document} % начало документа

\begin{titlepage}
\noindent
\begin{minipage}{0.05\textwidth}
\includegraphics[scale=0.4]{img/01.png}
\end{minipage}
\hfill
\begin{minipage}{0.85\textwidth}
\raggedleft

\begin{center}
\fontsize{12pt}{0.3\baselineskip}\selectfont \textbf{Министерство науки и высшего образования Российской Федерации \\ Федеральное государственное бюджетное образовательное учреждение \\ высшего образования \\ <<Московский государственный технический университет \\ имени Н.Э. Баумана \\ (национальный исследовательский университет)>> \\ (МГТУ им. Н.Э. Баумана)}
\end{center}
\end{minipage}

\begin{center}
\fontsize{12pt}{0.1\baselineskip}\selectfont
\noindent\makebox[\linewidth]{\rule{\textwidth}{4pt}} \makebox[\linewidth]{\rule{\textwidth}{1pt}}
\end{center}

\begin{flushleft}
\fontsize{12pt}{0.8\baselineskip}
\selectfont
ФАКУЛЬТЕТ \uline{<<\textbf{Фундаментальные науки}>>}
			
КАФЕДРА \hspace{4mm} \uline{<<\textbf{Прикладная математика}>>}
\end{flushleft}

\vspace{3mm}

\begin{center}
	\begin{large}
		\textbf{ДОМАШНЕЕ ЗАДАНИЕ \\ ПО ПРЕДМЕТУ:}
	\end{large}
\end{center}

\begin{center}
\begin{large}		
\uline{МЕТОДЫ ЧИСЛЕННОГО РЕШЕНИЯ}
	
\uline{ЗАДАЧ ЛИНЕЙНОЙ АЛГЕБРЫ}

\uline{Вариант №3}	
\end{large}
\end{center}

\vfill

\begin{flushright}
Выполнил: \\
студент группы ФН2-32М \\
Матвеев Михаил \\
\vspace{5mm}
Проверил: \\
Родин А. С. \\
\end{flushright}

\vfill

\begin{center}
\normalsize{Москва, 2020}
\end{center}

\end{titlepage}

\newpage

\tableofcontents

\newpage

\section{Постановка домашнего задания}

Нужно сформировать матрицу размером 10х10 по следующему принципу. В качестве базовой матрицы берется известная матрица, которая получается после дискретизации одномерного оператора Лапласа методом конечных разностей или методом конечных элементов на равномерной сетке:

\begin{equation*}
A_0 = 
\begin{pmatrix}
2 & -1 & 0 & 0 & \cdots & \cdots & \cdots & \cdots & 0 \\
-1 & 2 & -1 & 0 & \cdots & \cdots & \cdots & \cdots & 0 \\
0 & -1 & 2 & -1 & \cdots & \cdots & \cdots & \cdots & 0 \\
 &  &  &  & \ddots &  &  &  &  \\
0 & \cdots & \cdots & \cdots & \cdots & -1 & 2 & -1 & 0 \\
0 & \cdots & \cdots & \cdots & \cdots & 0 & -1 & 2 & -1 \\
0 & \cdots & \cdots & \cdots & \cdots & 0 & 0 & 2 & -1 \\
\end{pmatrix}
\end{equation*}


Для данной матрицы известны аналитические формулы для собственных значений (n=10):

\begin{equation}\label{eigenval_orig}
\lambda^{0}_j = 2(1-\cos(\frac{\pi j}{n + 1})), \quad j = 1, \ldots, n
\end{equation}

и компонент собственных векторов (вектора имеют 2-норму равную 1):

\begin{equation}\label{eigenvec_orig}
z^{0}_j(k) = \sqrt{\frac{2}{n + 1}}\sin(\frac{\pi j k}{n + 1}), \quad k = 1, \ldots, n
\end{equation}

Итоговая матрица получается по формулам:

\begin{equation*}
A = A_0 + \delta A,
\end{equation*}

\begin{equation*}
\delta A_{ij} = 
\left\{\begin{matrix}
\dfrac{c}{i + j}, & i \neq j \\ 
0 & i = j
\end{matrix}\right.,
\end{equation*}

\begin{equation*}
c = \frac{N_{var}}{N_{var} + 1}\varepsilon,
\end{equation*}

где $N_{var}$ - номер варианта (совпадает с номером студента в списке в журнале группы), $\varepsilon$  - параметр, значение которого задаётся далее.

\newpage

\section{Задача №1}

\subsection{Постановка задачи}
Взять матрицу A  для значения $\varepsilon = 0.1$ , убрать последний столбец и сформировать из первых 9 столбцов матрицу $\hat{A}$  размера 10х9. Решить линейную задачу наименьших квадратов для вектора невязки
\begin{equation*}
r = \hat{A} x - b,
\end{equation*}

где вектор b размерности 10х1 нужно получить по следующему алгоритму: выбрать вектор $x_0$  размерности 9х1 и для него вычислить $b = \hat{A} x_0$.

Для решения поставленной задачи использовать QR разложение: для вариантов с четным номером использовать соответствующий алгоритм, основанный на методе вращений Гивенса, для вариантов с нечетным номером – алгоритм, основанный на методе отражений Хаусхолдера. 
После получения решения сделать оценку величины $\dfrac{\left \| x - x_0 \right \|_{2}} {\left \| x_0 \right \|_{2}}$.

\newpage

\subsection{Применяемые методы}

\subsubsection{Преобразования Хаусхолдера}

Преобразованием Хаусхолдера (или отражением) называется матрица вида
\begin{equation}\label{matrix_house}
P = I - 2 u u^{T},
\end{equation}
где вектор $u$ называется вектором Хаусхолдера, а его норма $\left \| u \right \|_2$ = 1. Матрица $P$ симметрична и ортогональна, она называется отражением, потому что вектор $Px$ является отражением вектора $x$ относительно плоскости, проходящей через 0 перпендикулярно к $u$. 

Пусть дан вектор $x$. Тогда легко найти отражение $P = I - 2uu^T$, аннулирующее в векторе $x$ все компоненты, кроме первой: $Px = \left [c, 0, \ldots,0  \right ]^T = c \cdot e_1$. Это можно сделать следующим образом. Имеем $Px = x - 2u(u^T x) = c \cdot e_1$, поэтому $u = \dfrac{1}{2(u^T x)}(x - ce_1)$, т.е. $u$ есть линейная комбинация векторов $x$ и $e_1$. Так как $\left \| x \right \|_2 = \left \| Px \right \|_2 = \left | c \right |$, то $u$ должен быть параллелен вектору $\tilde{u} = x \pm \left \| x \right \|_2 e_1$, откуда $u = \dfrac{\tilde{u}}{\left \| \tilde{u} \right \|_2}$. Воспользуемся формулой $\tilde{u} = x + sgn(x_1) e_1$, так как в этом случае не будет взаимного сокращения при вычислении первой компоненты в $\tilde{u}$. Итак, имеем

\begin{equation*}
\tilde{u} = 
\begin{bmatrix}
x_1 + sgn(x_1) \cdot \left \| x \right \|_2 \\ 
x_2 \\ 
\vdots \\
x_n
\end{bmatrix}
\quad
u = \dfrac{\tilde{u}}{\left \| \tilde{u} \right \|_2}
\end{equation*}

Мы будем записывать данное преобразование как $u = house(x)$. После вычисления вектора Хаусхолдера получим отражение Хаусхолдера по формуле (\ref{matrix_house}). Далее умножаем матрицу, которую мы и хотим разложить на матрицы Q и R, на матрицу P слева. Данное действие аннулирует поддиагональные элементы матрицы A. Приведём общий алгоритм QR - разложения, основанный на использовании отражений. 

\newpage

\begin{algorithm}[H] 
	\SetAlgoLined
	\SetKwInOut{Input}{input}
	\SetKwInOut{Output}{output}

	\Input{Матрица A[m x n]}
	\Output{Матрицы R и Q}
m, n = A.shape\;
Q = np.identity(m)\;
R = np.copy(A)\;
for{ i in range(min(m, n)):}{
	u = house(A[i:, i])\;
	P = I - $2uu^T$\;
	A = P @ A\;
	Q = Q @ P\;
	}
\end{algorithm}

Обсудим некоторые детали реализации метода. Для хранения матрицы $P_i$ достаточно запомнить лишь вектор $u_i$. Эта информация может храниться в столбце $i$ матрицы $A_i$. Таким образом, QR - разложение может быть записано на место матрицы $A$, причём $P_i$ хранится в виде вектора $u_i$ в поддиагональных позициях столбца $i$ матрицы $A$. Так как диагональные позиции заняты элементами $R_{ii}$, нужно создать дополнительный массив для хранения первых элементов векторов $u_i$.

Так как мы решаем задачу наименьших квадратов $min\left \| Ax-b \right \|_2$ с помощью разложения $A = QR$, решение мы получаем, решая систему уравнений
\begin{equation*}
x = R^{-1} Q^T b.
\end{equation*}
В случае неявного QR-разложения (когда Q хранится в факторизованной форме $P_1, \cdots, P_{n-1}$, а $P_i$ хранится в виде вектора $u_i$ в поддиагональных позициях столбца $i$ матрицы $A$) нужно вычислить вектор $Q^Tb$. Это делается следующим образом $Q^T b = P_n P_{n-1}\cdots P_1 b$, поэтому b нужно последовательно умножать на $P_1, P_2, \cdots, P_n$:

\newpage

\paragraph{Описание алгоритмов}

\subparagraph{Обычный метод}

for i in range(n): (цикл по строкам)
\begin{enumerate}
\item подготавливаем матрицу $P_i$ (единичная матрица размера m x m, где m - количество строк);
\item в начале каждой итерации выбираем столбец матрицы A (не весь столбец, а элемент на диагонали и элементы под диагональю) в качестве вектора x;
\item применяем к вектору x метод house для получения вектора Хаусхолдера размера m - i;
\item вычитаем из матрицы A внешнее произведение двух векторов для получения матрицы ${P}'$ размера m - i x m - i;
\item добавляем матрицу ${P}'$ в матрицу $P_i$;
\item перемножаем матрицы $P_i$ и R для получения матрицы $A_i$, у которой обнуляются поддиагональные элементы;
\item домножаем матрицу Q на матрицу $P_i$ (для получения в конце итоговой матрицы Q);
\end{enumerate}
end for

\subparagraph{Эффективный метод}


for i in range(n): (цикл по строкам)
\begin{enumerate}
\item в начале каждой итерации выбираем столбец матрицы A (не весь столбец, а элемент на диагонали и элементы под диагональю) в качестве вектора x;
\item применяем к вектору x метод house для получения вектора Хаусхолдера размера m - i;
\item вычитаем из матрицы A внешнее произведение двух векторов;
\item записываем вектор u в поддиагональные элементы матрицы A кроме первого элемента вектора u;
\item записываем первый элемент вектора u в отдельный вектор;
\end{enumerate}
end for

\newpage

\lstinputlisting[language=Python, linerange = {1 - 33}, basicstyle = \small]{../qr_methods.py}

\begin{enumerate}
\item np.identity - создание двумерного массива, у которого элементы на главной диагонали единицы, остальные элементы - нули;
\item np.outer - внешнее произведение двух векторов;
\item np.linalg.norm - евклидова норма вектора;
\end{enumerate}

\newpage

\subsection{Результаты расчётов}

Был взят случайный вектор
\begin{equation*}
x_0 = [0.6401, 0.2454, 0.5507, 0.3099, 0.5663, 0.7639, 0.9395, 0.1872, 0.2075]
\end{equation*}

Полученное решение совпадает с начальным условием, величина относительной погрешности $\dfrac{\left \| x - x_0 \right \|_{2}} {\left \| x_0 \right \|_{2}}$ = $5.0357e^{-16}$.

\newpage

\subsection{Код решения}

\lstinputlisting[language=Python, linerange = {7 - 16}, basicstyle = \small]{../qr_tasks.py}

\lstinputlisting[language=Python, linerange = {75 - 89}, basicstyle = \small]{../qr_tasks.py}

\begin{enumerate}
\item amnt deletable cols - количество удаляемых столбцов
\item assert - проверка на верное QR - разложение
\end{enumerate}

\newpage

\section{Задача №2}

\subsection{Постановка задачи}

Для матрицы $A$ найти все ее собственные значения ($\lambda_j, \quad j = 1,\ldots, 10$) и собственные вектора ($z_j$ , с 2-нормой равной 1) с помощью неявного QR-алгоритма со сдвигом для трех вариантов: $\varepsilon = 10^{-1}, 10^{-3}, 10^{-6}$. 

По итогам расчетов нужно сделать сводную таблицу, в которой указать следующие величины: $\left | \lambda_j - \lambda_j^{0}  \right |$ и $\left \| z_j - z_j^{0} \right \|$  для $j = 1,\ldots,10$.

\newpage

\subsection{Применяемые методы}

\subsubsection{Базовый итерационный QR-алгоритм}

\paragraph{Описание алгоритма}

Пока не будет выполнено условие критерия сходимости:
\begin{enumerate}
\item выполняем QR - разложение $A_i = Q_i R_i$;
\item производим умножение $A_{i+1} = R_i Q_i$;
\end{enumerate}

Критерий сходимости - "Пока матрица А не станет достаточно близка к верхней треугольной матрице по своей структуре".

\lstinputlisting[language=Python, linerange = {18 - 27}, basicstyle = \small, title = Код алгоритма]{../qr_tasks.py}

\newpage

\subsubsection{Приведение матрицы к форме Хессенберга методом Хаусхолдера}
Дана квадратная симметричная матрица размера n x n;
for i in range(n - 2):
\begin{enumerate}
\item подготавливаем матрицу $Q$ (единичная матрица размера n x n);
\item в начале каждой итерации выбираем столбец матрицы A (не весь столбец, а элементы под первой диагональю) в качестве вектора x;
\item применяем к вектору x метод house для получения вектора Хаусхолдера размера n - i - 2 x n - i - 2;
\item вычитаем из единичной матрицы внешнее произведение двух векторов для получения матрицы ${P}'$ размера n - i - 2 x n - i - 2;
\item добавляем матрицу ${P}'$ в матрицу $Q$;
\item домножаем матрицу $A = Q A Q^T$;
\end{enumerate}
end for

\lstinputlisting[language=Python, linerange = {75 - 85}, basicstyle = \small, title = Код алгоритма]{../qr_methods.py}

\newpage

\subsubsection{Итерационный QR-алгоритм со сдвигом}

\paragraph{Описание алгоритма}

Приводим матрицу к форме Хессенберга до начала основного итерационного процесса;

for i in range(n, 0, -1):

Пока не будет выполнено условие критерия сходимости:

\begin{enumerate}
\item в качестве сдвига выбираем самый правый нижний элемент матрицы $\sigma = A_{i, i}$;
\item выполняем QR - разложение $A_i - \sigma \! E= Q_i R_i$;
\item производим умножение $A_{i+1} = R_i Q_i + \sigma \! E$;
\item если выполняется условие критерия сходимости, то правый нижний элемент матрицы принимается за собственное значение, сохраняется, а матрица размера A[i, i] лишается самой нижней строки и самого правого столбца;
\end{enumerate}
endfor

Критерий сходимости - "Пока самая нижняя строка матрицы А (за исключением правого элемента) и самый правый столбец (за исключением нижнего элемента) не станут близки к нулю".

\lstinputlisting[language=Python, linerange = {36 - 52}, basicstyle = \small, title = Код алгоритма]{../qr_tasks.py}

\newpage

\subsubsection{Метод Гивенса для неявного QR - алгоритма со сдвигом}

\paragraph{Описание алгоритма}

for i in range(n - 1):

\begin{enumerate}
\item подготавливаем единичную матрицу Q размера n x n;
\item в качестве сдвига выбираем самый правый нижний элемент матрицы $\sigma = A_{i, i}$;
\item вычисляем
\begin{equation*}
c = \dfrac{A_{i, i} - \sigma}{\sqrt{(A_{i, i} - \sigma)^2 + A_{i + 1, i}}} \quad
s = \dfrac{A_{i + 1, i}}{\sqrt{(A_{i, i} - \sigma)^2 + A_{i + 1, i}}};
\end{equation*}
\item вводим матрицу 
\begin{equation*}
\begin{pmatrix}
c & s \\
-s & c \\
\end{pmatrix}
\end{equation*}
в матрицу Q на позиции $[i, i] - [i + 1, i + 1]$;
\item домножаем матрицу $A = Q^T A Q$;
\end{enumerate}
endfor

\lstinputlisting[language=Python, linerange = {50 - 61}, basicstyle = \small, title = Код алгоритма]{../qr_methods.py}

\newpage

\subsubsection{Неявный QR-алгоритм со сдвигом}

\paragraph{Описание алгоритма}

Приводим матрицу к форме Хессенберга до начала основного итерационного процесса;

for i in range(n, 0, -1):

Пока не будет выполнено условие критерия сходимости:

\begin{itemize}
\item применяем к матрице Q метод Гивенса, описанный ранее;	
\item если выполняется условие критерия сходимости, то правый нижний элемент матрицы принимается за собственное значение, сохраняется, а матрица размера $A_{i, i}$ лишается самой нижней строки и самого правого столбца;
\end{itemize}
endfor

Критерий сходимости - "Пока самая нижняя строка матрицы А (за исключением правого элемента) и самый правый столбец (за исключением нижнего элемента) не станут близки к нулю".

\lstinputlisting[language=Python, linerange = {60 - 73}, basicstyle = \small, title = Код алгоритма]{../qr_tasks.py}

\newpage

\subsection{Результаты расчётов}

Были проведены расчёты для трёх параметров $\varepsilon = [1e-1, 1e-3, 1e-6]$, выведем четыре таблицы результатов: 
\begin{itemize}
\item таблица количества итераций, в которой самая первая строка - самое первое найденное собственное значение, и так далее по порядку;
\item отсортированная таблица количества итераций, в которой самая первая строка - минимальное собственное значение, и далее по возрастанию;
\item таблица $\left | \lambda_j - \lambda_j^{0}  \right |$, где в качестве $\lambda_j^0$ берётся (\ref{eigenval_orig});
\item таблица $\left \| z_j - z_j^{0} \right \|$  для $j = 1,\ldots,10$, где в качестве $z_j^{0}$ берётся (\ref{eigenvec_orig});
\end{itemize}

\begin{table}[h]
\centering
\caption{Таблица итераций}
\label{table_iters}
\begin{tabular}{|l|r|r|r|}
\hline
{} &  1e-01 &  1e-03 &  1e-06 \\
\hline
0 &      6 &      8 &     14 \\
\hline
1 &      2 &      2 &      1 \\
\hline
2 &      5 &      3 &      3 \\
\hline
3 &      3 &      3 &      2 \\
\hline
4 &      5 &      4 &      2 \\
\hline
5 &      2 &      3 &      3 \\
\hline
6 &      2 &      3 &      4 \\
\hline
7 &      2 &      2 &      2 \\
\hline
8 &      2 &      2 &      2 \\
\hline
9 &      1 &      1 &      1 \\
\hline
\end{tabular}
\end{table}

\begin{table}[h]
\centering
\caption{Таблица итераций (отсортированная по возрастанию)}
\label{table_iters_sorted}
\begin{tabular}{|l|r|r|r|}
\hline
{} &  1e-01 &  1e-03 &  1e-06 \\
\hline
0 &      5 &      3 &      3 \\
\hline
1 &      2 &      4 &      2 \\
\hline
2 &      2 &      3 &      3 \\
\hline
3 &      2 &      2 &      2 \\
\hline
4 &      1 &      1 &      1 \\
\hline
5 &      2 &      2 &      2 \\
\hline
6 &      3 &      3 &      4 \\
\hline
7 &      5 &      3 &      2 \\
\hline
8 &      2 &      2 &      1 \\
\hline
9 &      6 &      8 &     14 \\
\hline
\end{tabular}
\end{table}

\newpage

\begin{table}
\centering
\caption{Таблица абсолютных погрешностей для собственных значений}
\label{table_eigenvals}
\begin{tabular}{|l|l|l|l|}
\hline
{} &      1e-01 &      1e-03 &      1e-06 \\
\hline
0 &  7.103e-02 &  7.995e-04 &  8.003e-07 \\
\hline
1 &  1.961e-02 &  1.689e-04 &  1.687e-07 \\
\hline
2 &  1.887e-02 &  1.503e-04 &  1.500e-07 \\
\hline
3 &  5.989e-04 &  1.782e-05 &  1.792e-08 \\
\hline
4 &  9.211e-03 &  9.795e-05 &  9.800e-08 \\
\hline
5 &  2.016e-02 &  1.985e-04 &  1.985e-07 \\
\hline
6 &  2.391e-02 &  2.370e-04 &  2.370e-07 \\
\hline
7 &  2.411e-02 &  2.427e-04 &  2.427e-07 \\
\hline
8 &  1.937e-02 &  1.986e-04 &  1.987e-07 \\
\hline
9 &  1.216e-02 &  1.262e-04 &  1.262e-07 \\
\hline
\end{tabular}
\end{table}


\begin{table}
\centering
\caption{Таблица норм разностей для собственных векторов}
\label{table_eigenvecs}
\begin{tabular}{|l|l|l|l|}
\hline
{} &      1e-01 &      1e-03 &      1e-06 \\
\hline
0 &  1.776e-01 &  1.545e-03 &  1.543e-06 \\
\hline
1 &  1.762e-01 &  1.500e-03 &  1.495e-06 \\
\hline
2 &  1.041e-01 &  9.317e-04 &  9.306e-07 \\
\hline
3 &  4.239e-02 &  3.703e-04 &  3.780e-07 \\
\hline
4 &  3.038e-02 &  2.838e-04 &  2.836e-07 \\
\hline
5 &  3.631e-02 &  3.567e-04 &  3.542e-07 \\
\hline
6 &  4.926e-02 &  4.896e-04 &  4.896e-07 \\
\hline
7 &  5.655e-02 &  5.748e-04 &  5.123e-07 \\
\hline
8 &  5.583e-02 &  5.841e-04 &  5.843e-07 \\
\hline
9 &  3.866e-02 &  4.129e-04 &  4.245e-07 \\
\hline
\end{tabular}
\end{table}

\newpage

\subsection{Код решения}

\lstinputlisting[language=Python, linerange = {116 - 126}, basicstyle = \small, title = Код алгоритма]{../qr_tasks.py}

\newpage

\section{Выводы}

Были решены поставленные задачи с помощью методов Хаусхолдера и Гивенса, программы были написаны на языке Python. Все алгоритмы кода были предоставлены с кратким описанием действий. 

Решение первой задачи показало, что после QR разложения ответ совпадает с изначально заданным $x_0$, что показано оценкой $\dfrac{\left \| x - x_0 \right \|_{2}} {\left \| x_0 \right \|_{2}}$.

Решение второй задачи показало, что собственные значения, как и собственные вектора, при уменьшении значения $\epsilon$ становятся ближе к известным аналитическим собственным значениям и векторам. 

\end{document}
